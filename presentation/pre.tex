\documentclass{sysupptthesis}

% 注:可以在模版里面关闭宽屏

\author{王小明}
\title{基于应力矩阵的编队缩放控制\cite{2019Stress}}
\subtitle{申请中山大学工学学士学位论文答辩报告}
\institute{中山大学~计算机学院(软件学院)} % 可选邮箱
\date{
    %\today
    二〇二二年五月
}

\begin{document}

\begin{frame}
    \titlepage
    \vspace*{-0.5cm}
    \begin{figure}[htpb]
        \centering
        \includegraphics[width=0.12\linewidth]{../image/template/sysu-logo.pdf}
    \end{figure}
\end{frame}

\begin{frame}
    \tableofcontents[sectionstyle=show,subsectionstyle=show/shaded/hide,subsubsectionstyle=show/shaded/hide]
\end{frame}

\section{选题背景}

\begin{frame}

    \begin{block}{要点一\footnote{引用一}}
        \begin{itemize}
            \item 条目一,每一行字不要太密
            \item 条目二
            \item 条目三
        \end{itemize}
    \end{block}

    \begin{block}{要点二}
        \begin{itemize}
            \item 条目一\footnote{引用二}
            \item 条目二
        \end{itemize}
    \end{block}

\end{frame}

\section{相关概念与面临的挑战}


\begin{frame}[allowframebreaks]{图像搭配单页说明}

    \begin{block}{生僻字测试}
        \begin{itemize}
            \item 华为昇腾异构处理器\footnote{\url{https://www.hisilicon.com/cn/products/Ascend}}
            \item 条目二
        \end{itemize}
    \end{block}

    \begin{figure}[!htb]
        \includegraphics[width=0.5\textwidth]{../image/chap04/illustration/hole.pdf}
        \caption{单张图像}
        \label{fig:hole}
    \end{figure}

    \newpage

    \begin{block}{要点一}
        \begin{itemize}
            \item 条目一
            \item 条目二
        \end{itemize}
    \end{block}

    \begin{figure} %文中的Grid-LSTM模型做的语义图像分割的例子
        \centering
        \includegraphics[width=.2\textwidth,height=.1\textwidth]{../image/chap04/example/2007_000799.jpg}
        \includegraphics[width=.2\textwidth,height=.1\textwidth]{../image/chap04/example/2007_002094.jpg}
        \includegraphics[width=.2\textwidth,height=.1\textwidth]{../image/chap04/example/2007_004483.jpg}
        \includegraphics[width=.2\textwidth,height=.1\textwidth]{../image/chap04/example/2007_003194.jpg}
        \\
        \includegraphics[width=.2\textwidth,height=.1\textwidth]{../image/chap04/example/2007_000799.pdf}
        \includegraphics[width=.2\textwidth,height=.1\textwidth]{../image/chap04/example/2007_002094.pdf}
        \includegraphics[width=.2\textwidth,height=.1\textwidth]{../image/chap04/example/2007_004483.pdf}
        \includegraphics[width=.2\textwidth,height=.1\textwidth]{../image/chap04/example/2007_003194.pdf}
        \caption{并排的多张图像}
        \label{fig:multi-image-example1}
    \end{figure}

\end{frame}

\begin{frame}{单页大图}

    \begin{figure}
        \includegraphics[width=0.7\textwidth]{../image/chap03/overleaf-example.jpg}
        \caption{Overleaf使用例子,这里的描述可以长一些}
        \label{fig:overleaf-example}
    \end{figure}

\end{frame}

\section{算法设计}

\begin{frame}[allowframebreaks]{各种环境测试}
    \defination{定义示例}{这是定义}

    \theorem{定理示例}{这是定理}

    \suppose{假设示例}{这是假设}

    \lemma{引理示例}{这是引理}

    \begin{block}{再试试公式}
        \begin{itemize}
            \item 看下面!
        \end{itemize}
    \end{block}

    \begin{equation}
        e^{i\pi}+1=0
    \end{equation}

\end{frame}

\section{实验与结果}

\begin{frame}

    \begin{block}{最后还有表格}
        \begin{itemize}
            \item 往下看!
        \end{itemize}
    \end{block}

    \begin{table}
        \begin{tabular}{ccc}
            \hline
            姓名       & 学号 & 性别   \\
            \hline
            Steve Jobs & 001  & Male   \\
            Bill Gates & 002  & Female \\
            \hline
        \end{tabular}
        \caption{表格示例,乱写的}
        \label{fig:table-example}
    \end{table}

\end{frame}

\section{总结与展望}

\begin{frame}

    \begin{block}{要点一\footnote{引用一}}
        \begin{itemize}
            \item 条目一
            \item 条目二
        \end{itemize}
    \end{block}

    \begin{block}{要点二}
        \begin{itemize}
            \item 条目一\footnote{引用二}
            \item 条目二
            \item 条目三
        \end{itemize}
    \end{block}

\end{frame}

\section{参考文献}
\begin{frame}[allowframebreaks]
    \makereferences
\end{frame}

\section{Q \& A}

\begin{frame}

    \begin{block}{Questions?}
    \end{block}

\end{frame}

\section*{致谢}
\begin{frame}
    \begin{center}
        {\Huge\calligra Thanks!}
    \end{center}
\end{frame}

\end{document}