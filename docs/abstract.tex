% !TeX root = ../main.tex

\sysusetup{
  keywords  = {本科毕业论文,\LaTeX\ 模板,中山大学},
  keywords* = {undergraduate thesis, \LaTeX\ template, Sun Yat-Sen University},
  keywords** = {Mémoire de bachelor, L'Abstract, Mots-clés},
}

\begin{abstract}
  摘要应概括论文的主要信息,应具有独立性和自含性,即不阅读论文的全文,就能获得必要的信息。摘要内容一般应包括研究目的、内容、方法、成果和结论,要突出论文的创造性成果或新见解,不要与绪论相混淆。语言力求精练、准确,以300-500字为宜。关键词是供检索用的主题词条,应体现论文特色,具有语义性,在论文中有明确的出处,并应尽量采用《汉语主题词表》或各专业主题词表提供的规范词。关键词与摘要应在同一页,在摘要的下方另起一行注明,一般列3-5个,按词条的外延层次排列(外延大的排在前面)。

\end{abstract}

\begin{abstract*}
  The content of the English abstract is the same as the Chinese abstract, 250-400 content words are appropriate. Start another line below the abstract to indicate English keywords (Keywords 3-5).
\end{abstract*}

% 法语(第三种语言)摘要,本科不需要就全部注释掉即可,研究生请直接注释
%\begin{abstract**}
%  La longeur de l'abstract français doit faire référence à l'abstract chinois. Le titre de l'abstract français est «ABSTRACT». La première lettre de chaque mot-clé doit être en gros, et les mots-clés doivent être séparer par des virgules et espaces. Les mots-clés de chaque langue doivent être correspondants les uns aux autres.
%\end{abstract**}
